


\documentclass[11pt]{article}
\usepackage{fullpage}
\usepackage{amsthm}
\usepackage{amsthm,amsmath,amsfonts,amssymb,amstext}
\usepackage{latexsym,ifthen,url,rotating}
\usepackage[usenames,dvipsnames]{color}


% --- -----------------------------------------------------------------
% --- Document-specific definitions.
% --- -----------------------------------------------------------------
\newtheorem{definition}{Definition}

\newcommand{\concat}{{\,\|\,}}
\newcommand{\bits}{\{0,1\}}
\newcommand{\Range}{{\mathrm{Range}}}
\newcommand{\A}{{\mathcal{A}}}

% --- -----------------------------------------------------------------
% --- The document starts here.
% --- -----------------------------------------------------------------
\begin{document}
%\maketitle

\noindent CS513: Design and Analysis of Data Structures and Algorithms \\
Group: Stewart Smith and Kaushal Parikh\\

\begin{center}
    \LARGE{\textbf{Homework 2}}\\
\end{center}

\vspace{.1in}


\begin{enumerate}

    \item \\
        T(n) = 2T(n/2)+\log n
        f(n) = \log n \\
        a = 2, b = 2 \\
        a > \log _2 2 \\
        $ via Master Theorem \\$
        T(n) = \Theta(n ^ {(\log _b a)}) = \Theta(n)\\
    \item \\
    \item \\
        $Suppose: $ T(n) = n ^{1 - 2 ^ {-\log{\log{n}}}} + n\log{\log{n}}\\
        $Prove: $ T(n^2) = nT(n) + n^2\\
        = n(n ^{1 - 2 ^ {-\log{\log{n}}}} + n\log{\log{n}}) + n^2\\
        = n ^{2 - 2 ^ {1-(1+\log{\log{n}})}} + n^2(1+\log{\log{n}})\\
        = n ^{2 - 2 ^ {1-\log{\log{n^2}}}} + n^2\log{\log{n^2}}\\
        = (n^2)^{2 - 2 ^ {1-\log{\log{n^2}}}} + n^2\log{\log{n^2}}\\
        = T(n^2)\\
    \item \\
    \item \\
    \item Originally the rectangular array is given to us with it's row in ascending order\\
          We then take each of the columns and sort them.\\
          \begin{tabbing}
          Lemma: The smallest  item in a row is $>$ the smallest item in the previous row\\ \\
          Proof:\\$a b c \\  
           d e f \\ 
           g h i \\ $
                  \\1) If e is the lightest element in row 2 we know that d $<$ e
                  \\2) Thus when we sort row 1, we will have d or something $<$ d be the lighest element
                  \\3) This shows us that our lemma holds true\\
          \end{tabbing}
          \\We know that this works for the first row, and so by induction, it will also work for all subsequent rows\\

    \item Algorithm:
        \begin{verbatim}
                    HamC(G):
                        for each pair of neighbors(n1, n2):
                            if HamP(n1, n2, G):
                                return true
                        return false
                    \end{verbatim}

                    This approach will tell us if our graph has an hamiltonian circuit. \\
                    We are going through each pair of neighbors, which for a fully connect graph is going to be $n^2$ complexity\\
                    That makes our big O run time for the algorithm $n^2 * n^c = O(n^c)$, polynomial time\\
                    \\
                \item
                    Algorithm:
                    \begin{verbatim}
                findPath(n1, n2):
                    newPath
                    for each neighbor of n1 s:
                        if HamP(s, n1):
                            append s to newPath
                            findPath(s, n1)
               \end{verbatim}\\

               This approach will find us a hamiltonian path.\\
               The solution is to go to each neighbor and run the findPath algorithm on it again. \\
               This makes a new graph that consists only of the hamiltonian path.\\

       \end{enumerate}

       \end{document}

