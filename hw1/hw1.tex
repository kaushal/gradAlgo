


\documentclass[11pt]{article}
\usepackage{fullpage}
\usepackage{amsthm}
\usepackage{amsthm,amsmath,amsfonts,amssymb,amstext}
\usepackage{latexsym,ifthen,url,rotating}
\usepackage[usenames,dvipsnames]{color}


% --- -----------------------------------------------------------------
% --- Document-specific definitions.
% --- -----------------------------------------------------------------
\newtheorem{definition}{Definition}

\newcommand{\concat}{{\,\|\,}}
\newcommand{\bits}{\{0,1\}}
\newcommand{\Range}{{\mathrm{Range}}}
\newcommand{\A}{{\mathcal{A}}}

% --- -----------------------------------------------------------------
% --- The document starts here.
% --- -----------------------------------------------------------------
\begin{document}
%\maketitle
\sloppy

\noindent CS513: Design and Analysis of Data Structures and Algorithms \\
Group: Stewart Smith and Kaushal Parikh\\

\begin{center}
    \LARGE{\textbf{Homework 1}}\\
\end{center}

\vspace{.1in}

\begin{enumerate}

\item
    Height is at most n-1.  Add all nodes to the same side to achieve this height.\\
    Total number of nodes:\\
    1+2+4+8+ \cdots +2^k = \sum_{i=0}^k 2^i =  {2^{k+1}-1}
    $$n\leq 2^{d+1}-1\\
    $$n+1\leq 2^{d+1}\\
    log_2n \leq d+1\\
    $\lfloor {log_2n}\rfloor \leq d\\$

\item Prove log(n!) \in \Theta({nlogn})\\
    log(n!) = log(1) + log(2) + \cdots + log(n) \leq{log(n) + log(n) + \cdots  + log(n)}\\
    Assume: \sum_{i=1}^k ln(i)\\
    $Using Sterling's Approximation:\\$
    =\int _1 ^k \ln x dx\\
    =n\ln n \\
    log(n!) \in \Theta(nlogn)\\

\item (a)
    Base Case: for k=2 \sum_{i=1}^k i(i+1) = 8
    Assume: \sum_{i=1}^k i(i+1) = \frac1 3 k(k+1)(k+2)\\
    Prove:  \sum_{i=1}^{(k+1)} i(i+1) = \frac1 3 (k+1)(k+2)(k+3)\\
    \sum_{i=1}^{(k+1)} i(i+1) = \sum_{i=1}^k i(i+1) + (k+1)(k+2)\\
    = \frac1 3 k(k+1)(k+2) + (k+1)(k+2)\\
    = (k+1)(k+2) (\frac k3 + 1)\\
    = \frac1 3 (k+1)(k+2)(k+3)\\
    \\
    (b)\\
    Base Case: for k=1 \sum_{i=0}^k i2^i = 2
    Assume: \sum_{i=0}^k i2^i = (k-1)2^{k+1}+2\\
    Prove:  \sum_{i=0}^{k+1} i2^i = k2^{k+2}+2\\
    = \sum_{i=0}^k i2^i + (k+1)2^{k+1}\\
    = (k-1)2^{k+1}+2 + (k+1)2^{k+1}\\
    = 2^{k+1}(2k)\\
    = k2^{k+2} + 2\\
    \\
    (c)\\
    Base Case: for k=1 \sum_{i=0}^k \frac i{2^i} = \frac 1 2
    Assume: \sum_{i=0}^k \frac i{2^i} = 2- \frac {k+2}{2^k}\\
    Prove: \sum_{i=0}^{k+1} \frac i{2^i} = 2- \frac {k+3}{2^{k+1}}\\
    =\sum_{i=0}^k \frac i{2^i} + \frac {k+1}{2^{k+1}}\\
    = 2- \frac {k+2}{2^k} + \frac {k+1}{2^{k+1}}\\\\
    = 2- \frac {k+3}{2^{k+1}}\\
\item
    \begin{itemize}
        \item Assume the Minimal Spanning Tree is not unique, but a T1 and T2 exist and are ezual 
        \item There exists an edge e1 in T1, and not in T2 
        \item There exists an edge e2 in T2, and not in T1
        \item If T1 is an MST, then \ldots
    \end{itemize} 

\end{enumerate}

\end{document}

