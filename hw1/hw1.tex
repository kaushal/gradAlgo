


\documentclass[11pt]{article}
\usepackage{fullpage}
\usepackage{amsthm}
\usepackage{amsthm,amsmath,amsfonts,amssymb,amstext}
\usepackage{latexsym,ifthen,url,rotating}
\usepackage[usenames,dvipsnames]{color}


% --- -----------------------------------------------------------------
% --- Document-specific definitions.
% --- -----------------------------------------------------------------
\newtheorem{definition}{Definition}

\newcommand{\concat}{{\,\|\,}}
\newcommand{\bits}{\{0,1\}}
\newcommand{\Range}{{\mathrm{Range}}}
\newcommand{\A}{{\mathcal{A}}}

% --- -----------------------------------------------------------------
% --- The document starts here.
% --- -----------------------------------------------------------------
\begin{document}
%\maketitle
\sloppy

\noindent CS513: Design and Analysis of Data Structures and Algorithms \\
Group: Stewart Smith and Kaushal Parikh\\

\begin{center}
    \LARGE{\textbf{Homework 1}}\\
\end{center}

\vspace{.1in}

\begin{enumerate}


    \item Prove: A binary tree with n nodes has depth at least blog nc and at most n􀀀1. (Hint:: Show that if a binary tree has depth d and has n nodes, then n  2d+1 􀀀 \\
        $$n\leq 2^{d+1}-1\\
        $$n+1\leq 2^{d+1}\\
        log_2 \leq d+1\\

    \item Prove log(n!) \in \Theta({nlogn})\\
        log(n!) = log(1) + log(2) + \cdots + log(n) \leq{log(n) + log(n) + \cdots  + log(n)}\\
        log(n!) \in O(nlogn)\\

    \item Assume: \sum_{i=1}^k i(i+1) = \frac1 3 k(k+1)(k+2)\\
        Prove:  \sum_{i=1}^{(k+1)} i(i+1) = \frac1 3 (k+1)(k+2)(k+3)\\
        \sum_{i=1}^{(k+1)} i(i+1) = \sum_{i=1}^k i(i+1) + (k+1)(k+2)\\
        = \frac1 3 k(k+1)(k+2) + (k+1)(k+2)
    \item
        \begin{itemize}
            \item Assume the Minimal Spanning Tree is not unique, but a T1 and T2 exist and are ezual 
            \item There exists an edge e1 in T1, and not in T2 
            \item There exists an edge e2 in T2, and not in T1
            \item If T1 is an MST, then \ldots
        \end{itemize}
\end{enumerate}

\end{document}

